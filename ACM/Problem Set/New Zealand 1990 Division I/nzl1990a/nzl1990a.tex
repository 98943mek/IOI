\documentstyle{article}
\begin{document}
\section*{Problem A: Bumpy Objects}

Consider objects such as these.  They are polygons, specified by the
coordinates of a centre of mass and their vertices.  In the figure,
centres of mass are shown as black squares.  The vertices will be
numbered consecutively anti-clockwise as shown.

An object can be rotated to stand stably if two vertices can be found
that can be joined by a straight line that does not intersect the
object, and, when this line is horizontal, the centre of mass lies
above the line and strictly between its endpoints.  There are
typically many stable positions and each is defined by one of these
lines known as its base line.  A base line, and its associated stable
position, is identified by the highest numbered vertex touched by that
line.

Write a program that will determine the stable position that has the
lowest numbered base line.  Thus the desired
base line for the
square is 2.  You may assume that the objects are possible, that is they
will be represented as non self-intersecting polygons, although they
may well be concave.

Successive lines of a data set will contain: a string of less than 20
characters identifying the object; the coordinates of the centre of
mass; and the coordinates of successive points terminated by two
zeroes (0 0), on one or more lines as necessary.  There may be
successive data sets (objects).  The end of data will be defined by
the string '\#'.

Output will consist of the identification string followed by the
number of the relevant base line. 

\subsection*{Sample input}

\begin{verbatim}
Square
2 2
1 1  3 1  3 3  1 3 0 0
#
\end{verbatim}

\subsection*{Sample output}

\begin{verbatim}
1                   22
Square               2
\end{verbatim}

\newpage

\section*{Problem B: The Dole Queue}

In a serious attempt to downsize (reduce) the dole queue, The New
National Green Labour Rhinoceros Party has decided on the following
strategy.  Every day all dole applicants will be placed in a large
circle, facing inwards.  Someone is arbitrarily chosen as number 1,
and the rest are numbered counter-clockwise up to $N$ (who will be
standing on 1's left).  Starting from 1 and moving counter-clockwise,
one labour official counts off k applicants, while another official
starts from $N$ and moves clockwise, counting $m$ applicants.  The two who
are chosen are then sent off for retraining; if both officials pick
the same person she (he) is sent off to become a politician.  Each
official then starts counting again at the next available person and
the process continues until no-one is left.  Note that the two victims
(sorry, trainees) leave the ring simultaneously, so it is possible for
one official to count a person already selected by the other official.

Write a program that will successively read in (in that order) the
three numbers ($N$, $k$ and $m$; $k, m > 0$, $0 < N < 20$) and determine
the order in which the applicants are sent off for retraining.  Each
set of three numbers will be on a separate line and the end of data
will be signalled by three zeroes (0 0 0).

For each triplet, output a single line of numbers specifying the order
in which people are chosen.  Each number should be in a field of 3
characters.  For pairs of numbers list the person chosen by the
counter-clockwise official first.  Separate successive pairs (or
singletons) by commas (but there should not be a trailing comma).
Example:

\subsection*{Sample input}

\begin{verbatim}
10 4 3
0 0 0
\end{verbatim}

\subsection*{Sample output}

\begin{verbatim}
__4__8,__9__5,__3__1,__2__6,_10,__7
\end{verbatim}

where \verb|_| represents a single space.

\newpage

\section*{Problem C: Loglan - A Logical Language}

Loglan is a synthetic speakable language designed to test
some of the fundamental problems of linguistics, such as
the Sapir Whorf hypothesis.  It is syntactically unambiguous,
culturally neutral and metaphysically parsimonious. What follows is
a gross over-simplification of an already very small grammar of
some 200 rules.

Loglan sentences consist of a series of words and names, separated
by spaces, and are terminated by a period (\verb|.|).  Loglan words all end
with a vowel; names, which are derived extra-linguistically, end with
a consonant.  Loglan words are divided into two classes - little words
which specify the structure of a sentence, and predicates which have
the form CCVCV or CVCCV where C represents a consonant and V represents
a vowel (see examples later).

The subset of Loglan that we are considering uses the following grammar:

\begin{verbatim}
A => a | e | i | o | u
MOD => ga | ge | gi | go | gu
BA => ba | be | bi | bo | bu 
DA => da | de | di | do | du 
LA => la | le | li | lo | lu 
NAM => {all names} 
PREDA => {all predicates}
<sentence> => <statement> | <predclaim> 
<predclaim> => <predname> BA <preds> | DA <preds> 
<preds> => <predstring> | <preds> A <predstring> 
<predname> => LA <predstring> | NAM 
<predstring> => PREDA | <predstring> PREDA 
<statement> => <predname> <verbpred> <predname> |
               <predname> <verbpred>
<verbpred> => MOD <predstring> 
\end{verbatim}

Write a program that will read a succession of strings and determine
whether or not they are correctly formed Loglan sentences.

Each Loglan sentence will start on a new line and will be terminated
by a period (\verb|.|).  The sentence may occupy more than one line and words
may be separated by more than one space.  The input will be terminated
by a line containing a single '\#'.  You can assume that all words will
be correctly formed.

Output will consist of one line for each sentence containing 
either `\verb|Good|' or `\verb|Bad!|'.

\subsection*{Sample input}

\begin{verbatim}
la mutce bunbo mrenu bi ditca.
la fumna bi le mrenu.
djan ga vedma le negro ketpi.
#
\end{verbatim}

\subsection*{Sample output}

\begin{verbatim}
Good
Bad!
Good
\end{verbatim}

\newpage

\section*{Problem D: No Rectangles}

Consider a grid.  We wish to mark k
intersections in each of $n$ rows and $n$ columns in such a way that
no 4 of the selected intersections form a rectangle with sides
parallel to the grid.

It can easily be shown that for any given value of $k$, $k^2 - k +
1$ is a lower bound on the value of $n$, and it can be shown further
that n need never be larger than this.

Input will consist of lines each containing a value of $k$. 
Value of $n$ can be counted using a formula above. Write a program
that will find a solution to this problem for the given $k$.
The last line will contain 0.

Output will consist of $n$ lines of $k$ points indicating the selected
points on that line.

\subsection*{Sample input}

\begin{verbatim}
  2
  0
\end{verbatim}

\subsection*{Sample output}

\begin{verbatim}
  1  2
  1  3
  2  3
\end{verbatim}

\newpage

\section*{Problem E: Ugly Numbers}

Ugly numbers are numbers whose only prime factors are 2, 3 or 5.
The sequence

\[1, 2, 3, 4, 5, 6, 8, 9, 10, 12, 15,\] 

shows the first 11 ugly numbers.  By convention, 1 is included.

Write a program to find and print the $N$'th ugly number. The input
will consist of lines containing the numbers of $N$. The number on the last
line will be 0. Output should consist of lines as shown below, 
with \verb|<number>| replaced by the number computed.

\subsection*{Sample input}

\begin{verbatim}
1500
0
\end{verbatim}

\subsection*{Sample output}

\begin{verbatim}
The 1500'th ugly number is <number>.
\end{verbatim}

\newpage

\section*{Problem F: Polygons}

Given two polygons, they may or may not overlap.  If they do
overlap, they will do so to differing degrees and in different ways. 
Write a program that will read in the coordinates of the corners of two
polygons and calculate the `exclusive or' of the two areas,
that is the area that is bounded by exactly one of the polygons.  

Input will consist of pairs of lines each containing the number of
vertices of the polygon, followed by that many pairs of integers
representing the x,y coordinates of the corners in a clockwise
direction. All the coordinates will be positive integers less than 100.
For each pair of polygons (pair of lines in the data file), your
program should print out the desired area correct to two decimal places.
The input will end with a line containing a zero (0).

Output will consist of lines each containing the desired area written
as a succession of eight (8) digit fields with two (2) digits after the 
decimal point.

\subsection*{Sample input}

\begin{verbatim}
3  5 5  8 1  2 3
3  5 5  8 1  2 3
4  1 2  1 4  5 4  5 2
6  6 3  8 2  8 1  4 1  4 2  5 3
0
\end{verbatim}

\subsection*{Sample output}

\begin{verbatim}
____0.00
___13.50
\end{verbatim}

where \verb|_| represents a single space.

\newpage

\section*{Problem G: Street Numbers}

A computer programmer lives in a street with houses numbered
consecutively (from 1) down one side of the street.  Every evening she walks
her dog by leaving her house and randomly turning left or right and
walking to the end of the street and back.  One night she adds up the street
numbers of the houses she passes (excluding her own).  The next time she
walks the other way she repeats this and finds, to her astonishment,
that the two sums are the same.  Although this is determined in part
by her house number and in part by the number of houses in the street,
she nevertheless feels that this is a desirable property for her house to
have and decides that all her subsequent houses should exhibit it.

Write a program to find pairs of numbers that satisfy this condition.
To start your list the first two pairs are: (house number, last number):

\begin{verbatim}
         6         8
        35        49
\end{verbatim}

There is no input for this program.  Output will consist of 10 lines
each containing a pair of numbers in increasing order,
each printed right justified in a field of width 10 (as shown above).

\newpage

\section*{Problem H: Telephone Tangles}

A large company wishes to monitor the cost of phone calls made by its
personnel.  To achieve this the PABX logs, for each call, the number
called (a string of up to 15 digits) and the duration in minutes.
Write a program to process this data and produce a report specifying
each call and its cost, based on standard Telecom charges.

International (IDD) numbers start with two zeroes (00) followed by a country
code (1--3 digits) followed by a subscriber's number (4--10 digits).
National (STD) calls start with one zero (0) followed by an area code
(1--5 digits) followed by the subscriber's number (4--7 digits).  The
price of a call is determined by its destination and its duration.
Local calls start with any digit other than 0 and are free.

Input will be in two parts.  The first part will be a table of IDD and STD
codes, localities and prices as follows:

\begin{verbatim}
Code_Locality name$price in cents per minute
\end{verbatim}

where \verb|_| represents a space.  Locality names are 25 characters
or less.  This section is terminated by a line containing 6 zeroes (000000).

The second part contains the log and will consist of a series of
lines, one for each call, containing the number dialled and the
duration.  The file will be terminated a line containing a single '\#'.
The numbers will not necessarily be tabulated, although there will be
at least one space between them.  Telephone numbers will not be
ambiguous.

Output will consist of the called number, the country or area called, the
subscriber's number, the duration, the cost per minute and the total cost
of the call, as shown below.  Local calls are costed at zero.  If the number
has an invalid code, list the area as ``Unknown'' and the cost as -1.00.

\subsection*{Sample input}

\begin{verbatim}
088925 Broadwood$81
03 Arrowtown$38
0061 Australia$140
000000
031526        22
0061853279  3
0889256287213   122
779760 1
002832769 5
#
\end{verbatim}

\subsection*{Sample output}

\begin{verbatim}
1               17                     51   56    62     69
031526          Arrowtown            1526   22  0.38   8.36
0061853279      Australia          853279    3  1.40   4.20
0889256287213   Broadwood         6287213  122  0.81  98.82
779760          Local              779670    1  0.00   0.00
002832769       Unknown                      5        -1.00
\end{verbatim}
\end{document}
